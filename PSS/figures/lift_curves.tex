\begin{figure}
  \setlength{\unitlength}{\textwidth}

  \begin{picture}(1,0.75)(0,0)
    % % %90
      % % % Parkinson Data 
      \put(0.035,0.5){\includegraphics[width=0.5\unitlength]{./FnP/lift_curve_sq.eps}}
      \put(0.495,0.5){\includegraphics[width=0.5\unitlength]{./FnP/lift_curve_075.eps}}
      \put(0.035,0.27){\includegraphics[width=0.5\unitlength]{./FnP/lift_curve_05.eps}}
      \put(0.495,0.27){\includegraphics[width=0.5\unitlength]{./FnP/lift_curve_025.eps}}
      \put(0.035,0.02){\includegraphics[width=0.5\unitlength]{./FnP/lift_curve_tri.eps}}
      
      
   
      
      
%      \put(0.23,0.00){ $\displaystyle\frac{c}{\rho\mathcal{A}U}$}
%      \put(0.73,0.00){ $\displaystyle\frac{c}{\rho\mathcal{A}U}$}

      \put(0.28,0.00){\ustar}
      \put(0.78,0.00){\massdamp}
      
      \put(0.01,0.405){$\displaystyle C_y$}
       \put(0.01,0.65){$\displaystyle C_y$}
      
      \put(-0.028,0.18){$\displaystyle\small\frac{P_{m}}{\rho \mathcal{A}U^3 }$}
      
      \put(0.106,0.705){\small(a)}
      \put(0.565,0.705){\small(b)}
      \put(0.106,0.475){\small(c)}
      \put(0.565,0.475){\small(d)}
      \put(0.093,0.225){\small(e)}
      

  \end{picture}

  \caption{Induced lift coefficient $C_y$ at different angles for selected cross sections. Data presented for cross sections, (a) square, (b) $\ratio=0.75$, (c) $\ratio=0.5$, (d) $\ratio=0.25$ and (e) triangle.}
  \label{fig:lift_curves}
\end{figure}