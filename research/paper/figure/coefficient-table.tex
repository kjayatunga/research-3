\begin{table}[ht]

\begin{center}
\setlength{\unitlength}{\textwidth}

\begin{tabular}{c c c c c} % centered columns (4 columns)
\hline\hline %inserts double horizontal lines
\\[0.2ex]
Case & $a_1$ & $a_3$ & $a_5$ & $a_7$ \\ [0.8ex] % inserts table 
%heading
\hline 
\\[0.8ex]% inserts single horizontal line
Re=200 (polynomial-1) & 1.97 & 71.3  &-6937 &-254943  \\[0.8ex]
Re=200 (polynomial-2)& 2.32 & 197.8 & 4301.7 & 30311.9 \\[0.8ex]% inserting body of the table
Re=22300 & 2.69 & 168 & 1670 & 59900 \\ [1ex] % [1ex] adds vertical space
\hline %inserts single line
\end{tabular}

\caption{Coefficient values used in the 7th order interpolation polynomial for high ($Re=22300$) and low ($Re=200$) Reynolds numbers. Polynomial-1 was used to predict the $C_y$ at $\theta \leq 7^{\circ}$ while polynomial-2 was used to predict the $C_y$ where $\theta>7^{\circ}$ in order to obtain a better fit. These data are used as input data to calculate the right-hand side of Eq. \ref{final_equation_motion} throughout this study.}
 
\label{table:cy-coefficients} % is used to refer this table in the text
\end{center}
\end{table}

