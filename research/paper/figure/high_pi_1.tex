\begin{figure}
  \setlength{\unitlength}{\textwidth}
\fbox{
        \begin{picture}(1,1.1)(0,0.4)

      % % % Parkinson Data 
      \put(0.1,1.1){\includegraphics[width=0.75\unitlength]{../FnP/gnuplot/displacement_high_pi_1.eps}}
      \put(0.1,0.76){\includegraphics[width=0.75\unitlength]{../FnP/gnuplot/velocity_high_pi_1.eps}}
      \put(0.1,0.42){\includegraphics[width=0.75\unitlength]{../FnP/gnuplot/mean_power_high_pi_1.eps}}
      
         \put(0.07,0.95){$\displaystyle\frac{V}{D}$}
         \put(0.07,1.3){$\displaystyle\frac{A}{D}$}
         \put(0.05,0.6){$\displaystyle\frac{P_{m}}{\rho \mathcal{A}U^3 }$}
         \put(0.5,0.4){$\massdamp$}



%      
      \put(0.189,1.415){\small(a)}
      \put(0.189,1.07){\small(b)}
      \put(0.189,0.73){\small(c)}
%  

      
    \end{picture}
}
 \caption{QSS data at high \massstiff \ levels. (a) displacement amplitude, (b) velocity amplitude and (c) mean power as a function of \massdamp. Data presented at four different combined mass-stiffness levels.\ $\massstiff=10 \ (\mstar=20,\ \ustar=40)$ \ ($\times$),\ $\massstiff=100 \ (\mstar=80,\ \ustar=50) \ (+)$,\ $\massstiff=500 \ (\mstar=220,\ \ustar=60)$ \ (\ding{117}) and \ $\massstiff=1000 \ (\mstar=400,\ \ustar=40) \ (\triangle)$.}
    \label{fig:high_pi_1}
\end{figure}

 %vspace{10cm}
