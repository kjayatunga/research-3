\begin{table}[ht]

\begin{center}
\setlength{\unitlength}{\textwidth}

\begin{tabular}{c c c c c} % centered columns (4 columns)
\hline\hline %inserts double horizontal lines
\\[0.2ex]
$\ratio$ & $a_1$ & $a_3$ & $a_5$ & $a_7$ \\ [0.8ex] % inserts table 
%heading
\hline 

\\[0.8ex]% inserts single horizontal line
$0$ &  -2.30617 & -269.075 & -59.2929 & 4.74389\\[0.8ex]
    & -5.08342 & -56.5390e & -160.505e & -105.773\\[0.8ex]
    &  4.40685 & 19.9213 & 22.8894 & 7.68556\\[1ex]


\\[0.8ex]% inserts single horizontal line
$0.25$ & -0.605146 & -19.4346 &-82.4463 & -94.4226\\[0.8ex]
      & 2.50538 & 9.91021  & 10.2712 & 3.94112 \\[1ex]

 \\[0.8ex]% inserting body of the table


 $0.5$ & 1.44734 & 4.83885  & -166.900e & -983.072 \\[0.8ex]% inserting body of the table
  & 1.51455e & 15.8476 & 52.5465 & 62.8067 \\ [1ex] % [1ex] adds vertical space
  
  \\[0.8ex]% inserting body of the table
  
   $0.75$ & 1.76938 & 35.2630 & -345.562 & -10072.7 \\[0.8ex]
          & 1.77553 & 43.0120 & 262.983 & 638.484 \\ [1ex]
          
          
  
  
\hline %inserts single line


\end{tabular}

\caption{Coefficient values used in the 7th order interpolation polynomial at $Re=200$. Data present for $\ratio=0-0.75$ at increments of $0.5$. Multiples polynomials were used to attain a better fit. The plot of the compound fit is presented in figure \ref{fig:lift_curves-hybrid}.} 
 
\label{table:cy-coefficients-hybrid} % is used to refer this table in the text
\end{center}
\end{table}

