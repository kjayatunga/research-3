\begin{figure}
  \setlength{\unitlength}{\textwidth}

        \begin{picture}(1,1.1)(0,0.35)

      % % % Parkinson Data 
      \put(0.1,1.1){\includegraphics[width=0.75\unitlength]{./chapter-cross-sections/fnp/surf-pres-tri-4.eps}}
      \put(0.1,0.737){\includegraphics[width=0.75\unitlength]{./chapter-cross-sections/fnp/surf-pres-tri-16.eps}}
      \put(0.1,0.38){\includegraphics[width=0.75\unitlength]{./chapter-cross-sections/fnp/surf-pres-tri-21.eps}}
     
      
      



%      
    \put(0.21,1.41){\small(a)}
     \put(0.21,1.05){\small(b)}
     \put(0.21,0.69){\small(c)}
\put(0.1,0.95){$\displaystyle P_{s}$}
\put(0.1,1.3){$\displaystyle P_{s}$}
\put(0.1,0.56){$\displaystyle P_{s}$}
\put(0.26,0.35){Relative destance from the leading edge}

      
    \end{picture}

    \caption{Surface pressure of top (\ding{83}) and bottom (\ding{117})  surfaces of the static triangular cross section at (a) $\theta=4^\circ$, (b) $\theta=16^\circ$ \ and (c) $\theta=21^\circ$ A clear pressure difference is visible between the surfaces. The top surface comparatively has more negative pressure where a lift is created which results in a negative $C_y$ at $4^\circ$ and reduces as $\theta$ \ is increased, while the vice versa occurs at the top surface.}
    \label{fig:surf_pres}
\end{figure}

 %vspace{10cm}
