\chapter{Influence of fluid dynamics of the system on the extracted power}

\section{Introduction}

Galloping occurs as a result of the pressure difference created due to the relative distance between the shear layers and the respective walls of the cross section at the top and bottom sides of the body. As discussed in section \ref{subsec:c_y and shear layers}, the instantaneous induced angle makes one of the separated shear layers closer to the wall creating a low pressure region. This pressure difference results in a traverse forcing (normal to the direction of the flow) which becomes in phase with the transverse velocity and sustains galloping. 

From equation \ref{eqn:power_alt} discussed in section \ref{subsec:ave_pow} it is clear that the power transferred from fluid to the body is a function of the induced forcing $F_y$ and the transverse velocity $\dot{y}$. The sign of the average power represents the direction of power transfer where the $+$ ve sign represents the power transfer from fluid to the body and $-$ ve being power transferring from body to the fluid. 

Thus, then  according to equation \ref{eqn:power_alt} it can be hypothesised that if there is a scenario where both high induced forcing and high transverse is present, higher power output could be achieved. This brings to the analysis of the $C_y$ vs. $theta$ curve.           

