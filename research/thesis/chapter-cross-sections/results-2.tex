\chapter{Influence of fluid dynamics of the system on the extracted power}

\section{Introduction}

Galloping occurs as a result of the pressure difference created due to the relative distance between the shear layers and the respective walls of the cross section at the top and bottom sides of the body. As discussed in section \ref{subsec:c_y and shear layers}, the instantaneous induced angle makes one of the separated shear layers closer to the wall creating a low pressure region. This pressure difference results in a traverse forcing (normal to the direction of the flow) which becomes in phase with the transverse velocity and sustains galloping. 

From equation \ref{eqn:power_alt} discussed in section \ref{subsec:ave_pow} it is clear that the power transferred from fluid to the body is a function of the induced forcing $F_y$ and the transverse velocity $\dot{y}$. The sign of the average power represents the direction of power transfer where the $+$ ve sign represents the power transfer from fluid to the body and $-$ ve being power transferring from body to the fluid. 

Thus, then  according to equation \ref{eqn:power_alt} it can be hypothesised that if there is a scenario where both high induced forcing and high transverse velocities are present, higher power output could be achieved. This brings to the analysis of the $C_y$ vs. $\theta$ curve. \citet{Luo1994}, showed that the afterbody of the cross section has a direct impact on $C_y$ vs. $\theta$ curve. One interesting observation of this study was that delaying the shear layer re-attachment results in higher peak induced force coefficient $C_y$ occurring at high induced angles (high transverse velocities).    

Therefore, it could be hypothesised that a higher power transfer could be obtained by delaying the shear layer re-attachment. 

Here, the influence of shear layer and its reattachment on the mean power is studied by introducing a cross section which is a hybrid of a square and a triangle. The cross section is transformed gradually by manipulating the ratio of two length scales.

The stationary forcing data are presented for each cross section followed by the QSS power curves. Based on the QSS power data, an optimum cross section for power extraction is identified. As a negative region on some $C_y$ vs. $\theta$ curves were observed, the underpinning reason for this region was investigated through an analysis of the surface pressure and flow velocity data. The results and discussion of this analysis are presented. Following this, a comparison is made between QSS and DNS mean power at on the cross section which provides an optimum mean power.       

A final summary is presented explaining the influence of the behaviour of the shear layer on mean power output and the preliminary design considerations to optimise the fluid mechanics to obtain an optimum power output. 

\section{Influence of the shear layers}

In a typical cross section which sustains galloping, the induced lift \cy\ increases with increasing induced angle $\theta$ until it reaches a maximum value of \cy. The lift force is then  



