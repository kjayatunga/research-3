\chapter{A review of the literature}

\section{Flow induced vibrations}

Research on flow induced vibration has been conducted over the past century. This has been of interest because of the mechanical failure caused by these vibrations on civil structures. A simple structure that is susceptible to flow induced vibrations are slender structures, such as cylinders mounted perpendicularly to a fluid stream. With regards to a slender body two common types of flow induced vibrations are
%\begin{enumarate}

%	\item Vortex Induced Vibrations (VIV)
%	\item Fluid-elastic galloping
%\end{enumarate}

\subsection{Cause of galloping}
Galloping is occurred due to the instantaneous lift caused by interaction of the shear layers of the body moving in a flowing fluid. The phenomenon could be explained using a square cross section as an example. 

\subsection{Qusasi-state theory}

According \cite{Paidoussis2010}, \cite{Glauert1919} provided a criterion for galloping by considering the auto-rotation of an aerofoil and \cite{DenHartog1956} has provided a theoretical explanation for iced electric transmission lines. However, the study by \cite{Parkinson1964} could be identified as the pioneering study of galloping. A weakly non-linear oscillator model was developed by them to predict the response of the system. Essentially the quasi-steady assumption was made to develop this theory assuming that the instantaneous lift force of the oscillating body is equal to that of the lift force generated by the same body at the same induced angle of attack at the fixed support scenario. A detailed explanation is provided in section \hilight{the background theory section}.

The oscillator equation was solved using the Krylov and Bogoliubov method. Details of this method would not be mentioned as it is not used in the present study to solve the oscillator equation. The results obtained form experiments carried out at $\reynoldsnumber=2200$ and a mass ratio around 1164 had a good agreement with the theoretical data which is shown in \hilight{Parkinson amplitude data the figure}.

  


    

     










