\chapter{A review of the literature}

\section{Flow induced vibrations}


\section{Fluid-elastic galloping}

Fluid-elastic galloping is one of the most commonly observable flow-induced vibration on a slender body. Since this phenomenon is most common in civil structure, such as buildings and iced-transmission lines, the term ``aeroelastic galloping" is commonly used as the body is immersed in air. However, this mechanism can occur on a slender body immersed in any Newtonian fluid, provided that the conditions to sustain the galloping mechanism are satisfied. This work is based on a general Newtonian flow,thus the term `` fluid-elastic galloping" is used throughout this thesis.
   

\subsection{Excitation of galloping}

When a bluff body is moving along the transverse direction of the fluid flow, it generates a force along the transverse direction. This force also known as the induced lift is a resultant of the velocity of the fluid and the motion of the body. When this body is integrated to an oscillating system (i.e. spring, mass and damper system), the induced lift becomes the periodic forcing of the system. Galloping is sustained when the induced lift is in phase with the motion of the body. This could be explained further by using a square cross section. 

% % put figure 

% % \subsection{}      




\subsection{Qusasi-state theory}

According \cite{Paidoussis2010}, \cite{Glauert1919} provided a criterion for galloping by considering the auto-rotation of an aerofoil and \cite{DenHartog1956} has provided a theoretical explanation for iced electric transmission lines. However, the study by \cite{Parkinson1964} could be identified as the pioneering study of galloping. A weakly non-linear oscillator model was developed by them to predict the response of the system. Essentially the quasi-steady assumption was made to develop this theory assuming that the instantaneous lift force of the oscillating body is equal to that of the lift force generated by the same body at the same induced angle of attack at the fixed support scenario.








The oscillator equation was solved using the Krylov and Bogoliubov method. Details of this method would not be mentioned as it is not used in the present study to solve the oscillator equation. The results obtained form experiments carried out at $\reynoldsnumber=2200$ and a mass ratio around 1164 had a good agreement with the theoretical data which is shown in \hilight{Parkinson amplitude data the figure}.

  


    

     










