\chapter{Introduction}

The review of published literature reveals that fluid-elastic galloping has a potential to be used as a mechanism for energy extraction. Thus, the following questions emerged. What are the optimum parameters for energy transfer in a galloping system? How do they influence galloping?

Over the years, VIV has been the popular research problem  studied on flow induced vibrations. As a result, the parameters used to describe VIV problems (i.e \mstar, $\zeta$ and \ustar) has been incorporated to describe galloping, which could be observed throughout the current literature \citep{Barrero-Gil2009,Barrero-Gil2010a,Parkinson1964}.

However, data presented using this classical VIV parameters mean harvested power in particular \citep{Barrero-Gil2010a}, does not provide a good collapse.

Therefore it is hypothesised that more suitable parameters which could provide a better collapse of power output data could be obtained from the relevant time scales of galloping. 

 