\chapter{Governing parameters of  fluid-elastic galloping}
\label{chap:goven_para}

\section{Introduction}

This chapter contains the formulation of non dimensional governing parameters namely, the combined mass-stiffness \massstiff \ and the combined mass-damping \massdamp \ and the results and discussion demonstrating the influence of them. These parameters are formulated by obtaining the relevant time-scales of the system followed by non-dimesnionlising the governing QSS oscillator equation.  

A comparison of Quasi-steady state data presented using the classical VIV parameters and the newly formulated \massstiff \ and \massdamp is presented and it is concluded that \massdamp provides a better collapse for velocity amplitude and mean power compared the classical reduced velocity (\ustar) particularly because unlike \ustar, \massdamp \ does not include a frequency component in it. This is followed by the presentation of QSS data and discussion on the influence of \massstiff \ and \massdamp \ on power, which concludes that the power transfer is a primary function of \massdamp \ and a weak function of \massstiff.

Following this, a comparison of the QSS data with Direct Numerical Simulations (DNS) is presented. This reveals that the power transfer of the DNS data is strongly influenced by both \massstiff \ and \massdamp. Further analysis reveals that there is a good agreement between QSS and DNS for velocity and power at substantially high \massstiff\. As \massstiff \ decreases, the deviation (between QSS simulations and DNS) increases. Power spectral analysis of the DNS data shows a significant response at the vortex shedding at low \massstiff. The relative strength was found out to be an inverse function of \massstiff, which provides a clear explanation for the deviation between QSS simulations and DNS data at low \massstiff. This is primarily due to the influence of vortex shedding where this effect is not accounted in the QSS model.


\section{Formulation of the non-dimensionalised parameters \massstiff \ and \massdamp }

The natural time scales of the system could be obtained by linearising the quasi-steady equation of motion. (Eq:\KJ{equation of motion}) and finding the eigenvalues. The non-linear terms of the forcing function are truncated and the equation of motion could be expressed as, 

\begin{equation}
\label{eqn:eom_linear}
m\ddot{y}{+}c\dot{y}{+}ky{=}\frac{1}{2}\rho U^2 \mathcal{A} a_1\left(\frac{\dot{y}}{U}\right),
\end{equation}

After combining the $\dot{y}$ terms and solving for eigenvalues the following solutions for the eigenvalues could be obtained. 

 \begin{equation}
 \label{eqn:eigs}
 \lambda_{1,2}= -\frac{1}{2}\frac{c-\frac{1}{2}\rho U\mathcal{A}a_1}{m}\pm\frac{1}{2}\sqrt{\left[\frac{c-\frac{1}{2}\rho U\mathcal{A}a_1}{(m)}\right]^2-4\frac{k}{m}}.
 \end{equation} 
 
 Galloping essentially occurs at low frequencies therefore it can be assumed that the spring is relevantly weak and therefore, $k \rightarrow 0$. Hence a single non-zero eigenvalue remains which is, 
  
  \begin{equation}
  \label{eqn:eigs_nospring}
  \lambda=-\frac{c-\frac{1}{2}\rho U\mathcal{A}a_1}{m}.
  \end{equation}
  
  Further, if it is assumed that the mechanical damping is weaker than the fluid dynamic forces on the body the non zero eigenvalue could be further simplified to,
  
 \begin{equation}
 \label{eqn:eigs_nospring_nodamp}
 \lambda=\frac{\frac{1}{2}\rho U\mathcal{A}a_1}{m}.
 \end{equation}  

In this representation $\lambda$ represents the inverse time scale of the motion of the body due to the effect of long-time fluid dynamic forces (or forced due to the induced velocity). This term could also be re-written and $\lambda$ could be expressed as 

\begin{equation}
\label{eqn:timescale}
\lambda = \frac{a_1}{m^*}\frac{U}{D}
\end{equation}

This form clearly shows the significant parameters that influences the inverse time scale of the system. $\partial C_Y / \partial \alpha $, the rate of change in the fluid dynamic force on the body, with respect to the induced angle of attack, is represented by $a_1$. $\frac{U}{D}$ represents the inverse advective time scale of the incoming flow, and the mass ratio is resented by \mstar. Increasing $a_1$ would result in a rapid change of the fluid dynamic force with a small change of the induced angle $\theta$, which is proportional to transverse velocity $\dot{y}$. It can be seen in equation \ref{eqn:timescale} that an increase of $a_{1}$ would result in an increase of the inverse time scale or decrease the response time of the body. In contrast the mass ratio has the opposite effect where an increase in \mstar will lead to a decrease in $\lambda$, since a heavier body (or a body with higher inertia) would have a slower response. 

In order to find the relevant dimensionless groups of the problem, the time scale formulated could be used to non-dimensionalise the equation of motion. The equation of motion presented in Equation \KJ{put final equation of motion} can be non-dimensionalised using the non dimensional time $\tau$, defined as $\tau=t(a_1/m^*)(U/D)$. The non-dimensional equation of motion could then be represented as, 

 \begin{equation}
 \label{eqn:eom_nondim}
 \ddot{Y} + \frac{m^{*2}}{a_1^2}\frac{kD^2}{mU^2}Y = \left(\frac{1}{2} - \frac{m^*}{a_1}\frac{cD}{mU}\right)\dot{Y} - \frac{a_1A_3}{m^{*2}}\dot{Y}^3 + \frac{a_1^3a_5}{m^{*4}}\dot{Y}^5 - \frac{a_1^5a_7}{m^{*6}}\dot{Y}^7.
 \end{equation}
 
 The equation could be further altered by regrouping the coefficients into non-dimenasional groups and could be expressed as, 
 
  \begin{equation}
  \label{eqn:eom_nondim_regroup}
  \ddot{Y} + \frac{4\pi^{2}m^{*2}}{U^{*2}a_1^2}Y = \left(\frac{1}{2} - \frac{c^*m^*}{a_1}\right)\dot{Y} - \frac{a_1A_3}{m^{*2}}\dot{Y}^3 + \frac{a_1^3a_5}{m^{*4}}\dot{Y}^5 - \frac{a_1^5a_7}{m^{*6}}\dot{Y}^7,
  \end{equation}  

\ustar is the reduced velocity which is the typical independent variable ussed in vortex-induced vibration studies. \cstar is the non-dimensional damping parameter which is expressed as $c^*=cD/mU$. 

By analysing equation \ref{eqn:eom_nondim_regroup} it is clear that five dimensionless parameters play a role in setting the response of the system. These are namely the stiffness, damping, mass ratio, the geometry and the Reynolds number. The stiffness is repented by the reduced velocity \ustar, the damping by \cstar and the mass ratio by \mstar. The geometry and the Reynolds number are represented by the coefficients $a_n$, of the polynomial fit to the $C_y$ curve. Using the natural time scales of the system, grouping of these non-dimensional parameters into two groups in the non-dimensional equation of motion, suggests that there are two groups that governs the response which are: $\Gamma_1 = 4\pi^2m^{*2}/U^{*2}a_1^2$ and $\Gamma_2 = c^*m^*/a_1$. $\Gamma_1$ could be described as a combined mass-stiffness, where $\Gamma_2$ could be expressed as a combined mass-damping parameter for a given geometry and a Reynolds number. It is assumed that the stiffness plays a minor role, $\Gamma_2$ seems more likely parameter to collapse the data. The wind tunnel data in the classic paper of galloping by \citep{Parkinson1964} adopted a parameter similar to $\Gamma_2$ to collapse the data. 

All of the quantities that formulate $\Gamma_1$ and $\Gamma_2$ except $a_1$ in theory, could be obtained before an experiment. However in order to obtain the value of $a_1$ static body experiments are required making it relatively difficult to obtain. Here, the \reynoldsnumber and the geometry remains constant and therefore multiplying $\Gamma_1$ with ${a_1}^2$ and $\Gamma_2$ with $a_1$ suitable parameters could be obtained, and formulate a mass-stiffness parameter $\massstiff =  4\pi^2m^{*2}/U^{*2}$, and a mass-damping parameter defined as $\massdamp = c^*m^*$.





    


