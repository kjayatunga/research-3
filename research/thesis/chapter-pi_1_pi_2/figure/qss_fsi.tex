\begin{figure}
  \setlength{\unitlength}{\textwidth}

        \begin{picture}(1,1.1)(0,0.35)

      % % % Parkinson Data 
      \put(0.1,1.1){\includegraphics[width=0.75\unitlength]{./chapter-pi_1_pi_2/FnP/gnuplot/fqss_fsi_displace.eps}}
      \put(0.1,0.737){\includegraphics[width=0.75\unitlength]{./chapter-pi_1_pi_2/FnP/gnuplot/qss_fsi_velocity.eps}}
      \put(0.1,0.38){\includegraphics[width=0.75\unitlength]{./chapter-pi_1_pi_2/FnP/gnuplot/qss_fsi_power.eps}}
      
      



%      
    \put(0.15,1.41){\small(a)}
     \put(0.15,1.05){\small(b)}
     \put(0.15,0.69){\small(c)}
\put(0.03,0.95){$\displaystyle\frac{V}{U}$}
\put(0.03,1.3){$\displaystyle\frac{A}{D}$}
\put(0.0,0.56){$\displaystyle\frac{P_{m}}{\rho \mathcal{A}U^3 }$}
\put(0.466,0.35){$\massdamp$}

      
    \end{picture}

    \caption{Comparison of data generated using the quasi-static model
      and full DNS simulations at (a) Displacement amplitude, (b)
      velocity amplitude and (c) dimensionless mean power as functions of
      \massdamp. Data were obtained at $\reynoldsnumber = 200$ at four
      values $\massstiff=10$ ($\mstar= 20.13$) (\ding{83}),
      $\massstiff=60$ ($\mstar =49.31$) (\ding{108}), $\massstiff=250$
      ($\mstar= 100.7$) ($\triangle$) and $\massstiff=1000$ ($\mstar=201.3$) (\ding{117}). The QSS data at $\massstiff=10$ \
      (\protect\dashedrule).}
    \label{fig:qss_fsi}
\end{figure}

 %vspace{10cm}
