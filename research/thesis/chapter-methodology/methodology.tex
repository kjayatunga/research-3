\chapter{Methodology and validation}

\section{Introduction}
 
A brief overview of the computational methods to perform the simulations to obtain the data in this thesis are are presented in this chapter. As the this particular study is not focused on developing computational methods but concentrated on understanding the physics of a body under the influence of fluid-elastic galloping, it should be noted that the overview provided in this chapter is quite abstract.

The flow of this chapter is as follows. \hilight{First the equations used to model the system are presented and discussed. Next, the a brief discussion of the direct numerical simulation is presented followed by the problem formulation and the discussion of the parameters used.} 

Finally, a series of validation data are presented and discussed to ensure the accuracy of the direct numerical simulations in order to ensure the confidence in the numerical predictions of this thesis.


\subsection{Parameters used}

The data in this project are mainly presented in two categories, high and low Reynolds numbers to compare results at laminar and turbulent range. One main objectives in this study was to capture the flow physics accurately using direct numerical simulations hence,major portion of the study was carried out in the laminar range where the flow is close to 2D. Although majority of data are focused on low Reynolds numbers, some data were presented using inputs from high Reynolds numbers to the QSS model to provide a comparison between high and low Reynolds numbers. $\reynoldsnumber=200$ was defined as the ``low" Reynolds number and $\reynoldsnumber=22300$ was defined as the high Reynolds number. Studies by \citet{tong2008} and \citet{sheard2009} reveals that the approximate value of 3-dimensional transition of the wake for a square cross section is $\reynoldsnumber=160$ and therefore, $\reynoldsnumber=200$ was selected to represent the low Reynolds number regime, also considering the fact that other numerical studies in the laminar regime have used this value for the Reynolds number \citep{Robertson2003,Joly2012}. The reason behind considering the flow regimes of a square cross section was the fact that the basic cross section being used in this study was a square. The selection of the value for the high Reynolds number was fairly was simple as it was the Reynolds number where the pioneering study of galloping \citet{Parkinson1964} provided the experimental input data ($C_y data$) for the QSS model. 

Stationary $C_y$ data at different angles of attack to be used as inputs to the QSS model, were obtained  for the low Reynolds number regime using direct numerical simulations. The average power was obtained by using equation \ref{power}, and the averaging was done over no less than 20 galloping periods. For the high \reynoldsnumber\ tests, predictions of power output at $\reynoldsnumber=22300$ were obtained using the coefficients for the $C_y$  curve from \citet{Parkinson1964}. The mass ratio $m^*$ was kept at 1163 for $\reynoldsnumber=22300$ (Similar to \citet{Parkinson1964}), $m^*=20$ for \reynoldsnumber=200 and $\ustar\geq 40$ similar to the parameters used in literature \citep{Robertson2003,Joly2012} to obtain a comparison with published work. These parameters were used throughout this study unless otherwise specified.

\section{Quasi-steady model}
\label{sec:QSS_model_methodology}

The quasi-steady state model discussed in section \ref{sec:QSS theory} was used to obtain oscillator response data. The quasi-steady state model has proven its ability to obtain accurate galloping response data (discussed in section \ref{sec:QSS theory}). Therefore, it enables to obtain large number of at the expense of a short computational time. The oscillator equation consist of spring, mass and damper oscillator expression with a $7^{th}$ order interpolation polynomial as the forcing function (equation \ref{final_equation_motion}).

\subsubsection{Solving the quasi-steady state equation}

The quasi-steady model being an ordinary differential equation could be solved using different solving methods. Some of the techniques include limit cycle oscillations, harmonic balance, cell mapping and numerical integration. \citet{Vio2007} showed that numerical integration provides accurate data. A fourth-order Runge-Kutta ODE solving scheme was used in solving the quasi-steady state oscillator equation. The built in `ode45' function in MATLAB was used primarily to solve the QSS equation while in some cases `ode15s' function was used when the equation became more stiff.


\section{Calculation of average power}

The ideal potential amount of harvested power output could be represented as the dissipated power due to mechanical damping before losses in any power take-off system are included. Thus the mean power output could be expressed as 


\begin{equation}
\label{power}
P_{m}=\frac{1}{T}\int_{0}^{T}(c\dot{y})\dot{y} dt,
\end{equation}
where $T$ is the period of integration and $c$ is the mechanical damping constant. 

The work done on the body by the fluid is equal to this quantity, defined as
\begin{equation}
\label{power_alt}
P_{m}=\frac{1}{T}\int_{0}^{T}F_y\dot{y} dt,
\end{equation}
where $F_y$ is the transverse (lift) force.

The two definitions of the mean power provide two vital interpretations of power transfer. Equation \ref{power} shows that the power is proportional to the mechanical damping and the magnitude of the transverse velocity. At first glance one may assume that the power could be increased by increasing damping. In a practical power extraction device, the significant component of damping would be due to the electrical generator and therefore, an increase in damping would be due to the increase of the load or in other words the electrical resistance. Yet this perception of damping is not quite accurate as very high damping would result in reducing the velocity amplitude which then, would not result in a higher energy output according to equation \ref{power}. In consequence, a balance need to be obtained where the damping is high, but not to the extent that it will adversely result by overly suppressing the motion of the body.  

On the other hand, equation \ref{power_alt} shows that a higher power is attained during situations where the transverse force $F_{y}$ and the transverse velocity are in phase. Hence, a simple increase in the magnitude of the force or the velocity is not satisfactory to attain a higher power transfer. Any increase in magnitude of either of the parameters (force or velocity) is linked to an increase in phase.   



\subsubsection{Direct numerical simulations (DNS)}

Direct numerical simulations were employed to obtain the stationary data to be used as inputs to the QSS model and to obtain  fluid-structure interaction (FSI) data to be compared with the QSS model at low Reynolds numbers. A high-order in-house build spectral element which simulates two-dimensional laminar flows was used to obtain the DNS data.   


\subsubsection{Boundary conditions}

The boundary conditions, regardless of the mesh  were common for all the simulations performed. A no-slip condition was applied to the cross section wall. This condition implied that the velocity is zero at the surface of the cross section. For stationary simulations a Dirichlet boundary condition and for FSI cases a time-dependent Dirichlet boundary condition was employed for the velocity on the inlet and lateral boundaries. A Dirichlet boundary condition should have a specified value for the variables \citep{kreyszig2010} in this case velocity. The time-dependent Dirichlet condition has to be implemented for the FSI cases to account for the accelerated reference frame attached to the cross section. Thus, the inlet boundary was set to $u=U$ and $v=-\dot{y}$ for FSI cases and $v=0$ for stationary cases, where $u,v$ are the velocities in the $x$ and $y$ directions, respectively.

A Neumann condition for the pressure (where the gradient of a property is specified \citet{tu2007}), where the normal gradient was calculated from the Navier--Stokes equations, was employed on the inlet, lateral and body surface \citep{gresho1987}, while a Dirichlet condition for the pressure ($p=0$ was enforced at the outlet. The details of the method can be found in \citet{Thompson2006,Thompson1996a}

 Although the physical validity of the outlet boundary condition is not quite true, this does not turn out to be a significant problem provided that the Reynolds numbers are low and the domain is sufficiently far away from the body.


\subsubsection{Spectral element method}
 
 To obtain DNS results an in-house build code was used. This code essentially solves the Naiver-Stokes equations in an accelerated reference frame. A three-step time-splitting scheme also known as a fractional step method was used for temporal discretisation. A predictor-corrector method was used for the FSI data where an elastically mounted body was involved. A description of the spectral element method in general can be found in \citet{karniadakis2005}. This code has been very well validated in a variety of fluid-structure interaction problems similar to that studied in the current study \citep{Leontini2007a,Griffith2011,Leontini2011,Leontini2013}. Therefore, a validation studies for the code, the method and the algorithms are not discussed in this thesis. 
 
\subsection{Convergence and validation studies}

\subsubsection{Domain size}

 For all cases, a rectangular domain was employed where the inlet was placed $20D$ from the centre of the body, while the outlet was situated $60D$ away from the centre of the body. The lateral boundaries were placed $20D$ away from the centre of the body.








%The computational domain consists of 751 quadrilateral macro elements where the majority of the elements were concentrated near the square section. A freestream condition was given to the inlet, top and bottom boundaries and the normal velocity gradient was set to zero at the outlet. A convergence study was performed by changing the order of the polynomial ($p$-refinement) at $U^*=40$ and $\reynoldsnumber=200$. A $9^{th}$ order polynomial together with a time step of $\Delta tU/D=0.001$ was sufficient to ensure an accuracy of $2\%$ with regards to amplitude of oscillation.










