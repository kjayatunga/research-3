\chapter{Methodology and validation}

\section{Introduction}
 
A brief overview of the computational methods to perform the simulations to obtain the data in this thesis are are presented in this chapter. As the this particular study is not focused on developing computational methods but concentrated on understanding the physics of a body under the influence of fluid-elastic galloping, it should be noted that the overview provided in this chapter is quite abstract.

The flow of this chapter is as follows. First the equations used to model the system are presented and discussed. Next, the a brief discussion of the direct numerical simulation is presented followed by the problem formulation and the discussion of the parameters used. 

Finally, a series of validation data are presented and discussed to ensure the accuracy of the direct numerical simulations in order to ensure the confidence in the numerical predictions of this thesis.

\section{Quasi-steady model}
\label{sec:QSS_model_methodology}

The quasi-steady state model discussed in section \ref{sec:QSS theory} was used to obtain oscillator response data. The quasi-steady state model has proven its ability to obtain accurate galloping response data (discussed in section \ref{sec:QSS theory}). Therefore, it enables to obtain large number of at the expense of a short computational time. The oscillator equation consist of spring, mass and damper oscillator expression with a $7^{th}$ order interpolation polynomial as the forcing function (equation \ref{final_equation_motion}).

\section{Calculation of average power}

The ideal potential amount of harvested power output could be represented as the dissipated power due to mechanical damping before losses in any power take-off system are included. Thus the mean power output could be expressed as 


\begin{equation}
\label{power}
P_{m}=\frac{1}{T}\int_{0}^{T}(c\dot{y})\dot{y} dt,
\end{equation}
where $T$ is the period of integration and $c$ is the mechanical damping constant. 

The work done on the body by the fluid is equal to this quantity, defined as
\begin{equation}
\label{power_alt}
P_{m}=\frac{1}{T}\int_{0}^{T}F_y\dot{y} dt,
\end{equation}
where $F_y$ is the transverse (lift) force.

The two definitions of the mean power provide two vital interpretations of power transfer. Equation \ref{power} shows that the power is proportional to the mechanical damping and the magnitude of the transverse velocity. At first glance one may assume that the power could be increased by increasing damping. In a practical power extraction device, the significant component of damping would be due to the electrical generator and therefore, an increase in damping would be due to the increase of the load or in other words the electrical resistance. Yet this perception of damping is not quite accurate as very high damping would result in reducing the velocity amplitude which then, would not result in a higher energy output according to equation \ref{power}. In consequence, a balance need to be obtained where the damping is high, but not to the extent that it will adversely result by overly suppressing the motion of the body.  

On the other hand, equation \ref{power_alt} shows that a higher power is attained during situations where the transverse force $F_{y}$ and the transverse velocity are in phase. Hence, a simple increase in the magnitude of the force or the velocity is not satisfactory to attain a higher power transfer. Any increase in magnitude of either of the parameters (force or velocity) is linked to an increase in phase.   

\section{Discretisation of the QSS model and solving method}















