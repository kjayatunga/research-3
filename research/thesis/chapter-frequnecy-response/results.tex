\chapter{Frequency response of the system}

\section{Introduction}

\section{Linear frequency of the system}

The eigenvalues of the linearised QSS model could be found in equation \ref{eqn:eigs}. The term under the square root (equation \ref{eqn:liner_freq}) of this equation can be used to express the frequency of the system provided that the term is complex. Hence, the frequency could be defined as the imaginary portion of this complex number


\begin{equation}
\label{eqn:liner_freq}
f = \sqrt{\left[\frac{c-\frac{1}{2}\rho U\mathcal{A}a_1}{(m)}\right]^2-4\frac{k}{(m)}}.
\end{equation}



% % % % % % % % %
By substituting \cstar, \mstar\ and \ustar equation \ref{eqn:liner_freq} could be non-dimensionalised as follows:

\begin{equation}
f = \sqrt{\left[c^*\left(\frac{U}{D}\right) - \frac{1}{2}\frac{a_1}{m^*}\left(\frac{U}{D}\right)\right]^2 - 4\left(\frac{U}{D}\right)^2\frac{2\pi}{U^*}}.
\end{equation}

This can then be rewritten as
\begin{equation}
f = \sqrt{\left(\frac{U}{D}\right)^2\left(c^*-\frac{a_1}{2m^*}\right)^2 - 4\left(\frac{U}{D}\right)^2\left(\frac{2\pi}{U^*}\right)^2}.
\end{equation}
By taking the factor of $U/D$ to the left-hand side
\begin{equation}
\frac{fD}{U} = \sqrt{\left(c^*-\frac{a_1}{2m^*}\right)^2 - 4\left(\frac{2\pi}{U^*}\right)^2}.
\end{equation}
Expanding terms gives
\begin{equation}
\frac{fD}{U} = \sqrt{c^{*2} - \frac{2c^*a_1}{2m^*} + \frac{a_1^2}{4m^{*2}} - \frac{16\pi^2}{U*^2}}.
\end{equation}
Multiplying through by $m*^2$ gives
\begin{equation}
\frac{fD}{U} = \sqrt{c^{*2}m^{*2} - c^*m^*a_1 + \frac{a_1^2}{4} - \frac{16\pi^2m{*^2}}{U^{*2}}}.
\end{equation}


By substituting \massstiff\ and \massdamp\ appropriately the expression of the linear frequency reduced to   
\begin{equation}
\label{eqn:linear_freq_final}
\frac{fD}{U} = \sqrt{\Pi_2^2 - \Pi_2a_1 + \frac{a_1^2}{4} - 4\Pi_1}.
\end{equation}

Thus, from equation \ref{eqn:linear_freq_final} the non-dimensionalised linear frequency of the system could be expressed from the newly formulated terms, \massstiff\ and \massdamp.



So, by setting $f=0$, the relationship between $\Pi_1$ and $\Pi_2$ at
the limit can be found.

\section{Finding the terminal velocity of the body when no frequency
	is predicted by equation 8}
	
	For very small $Pi_1$ where no frequency is predicted by equation 8,
	we can assume that the body quickly accelerates to a velocity where
	the lift force is blanced by the damping force. While the displacement
	is small, the spring force is basically negligible. Also, for the
	velocity to saturate (reach a constant value), we need only one
	nonlinear term in the equation, and so we retain only up to the cubic
	velocity term in the lift force to give
	\begin{equation}
	c\dot{y} = \frac{1}{2}\rho U^2\mathcal{A}\left[a_1\left(\frac{\dot{y}}{U}\right) + a_3\left(\frac{\dot{y}}{U}\right)^3\right].
	\end{equation}
	Rearranging and dividing by $\dot{y}$ gives
	\begin{equation}
	\left(\frac{1}{2}\rho U\mathcal{A}a_1 - c\right) + \frac{1}{2}\rho\frac{1}{U}\mathcal{A}a_3\dot{y}^2 = 0,
	\end{equation}
	So that the terminal velocity $\dot{y}$ is given by
	\begin{equation}
	\dot{y} = \pm\sqrt{-\frac{(1/2)\rho U^2\mathcal{A}a_1 - cU}{(1/2)\rho\mathcal{A}a_3}}.
	\end{equation}
	This can be written as
	\begin{equation}
	\dot{y} = \pm\sqrt{U^2\frac{a_1}{a_3} - U^2\frac{c}{(1/2)\rho U\mathcal{A}a_3}}
	\end{equation}
	or
	\begin{equation}
	\frac{\dot{y}}{U} = \pm\sqrt{\frac{a_1}{a_3} - \frac{c}{(1/2)\rho U\mathcal{A}a_3}}.
	\end{equation}
	
	Finally, the definition of $\Pi_2$ can be substituted into the last term to give
	\begin{equation}
	\frac{\dot{y}}{U} = \pm\sqrt{\frac{1}{a_3}(a_1 - 2\Pi_2)},
	\end{equation}
	hence the terminal, or maximum velocity is a function of $\Pi_2$ only.




