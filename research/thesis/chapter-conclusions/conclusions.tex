\chapter{Conclusions}

A fundamental study was carried out to explore the potential of obtaining useful energy from fluid-elastic galloping. This research was employed theoretical modelling and numerical simulations. The study was primarily on understanding the energy transfer between the fluid and the structure.  

Galloping being a phenomenon of fluid-structure interaction, two main objectives were identified during the selection of the scope of the research. These were understanding the underpinning structural parameters of the system and methods to optimise it for a better power output and understanding the fluid mechanics of the system and thereby attempting to optimise mean power output through manipulation of the fluid-dynamics of the system.

New governing dimensionless groups for galloping namely, \massstiff\ and \massdamp\ were formulated using the natural times-scales of the linearised quasi-steady state model. Data were obtained for a square cross section. The formulated dimensionless groups provided a good collapse for the predicted power output in comparison with the classical VIV parameters i.e. \ustar, \mstar\ and $\zeta$. The data collapse as a result of using the dimensionless groups reinforce the argument that the velocity amplitude of the system and the power transfer of the system do not depend on the natural frequency of the system over a large range of natural frequencies. Although equation \ref{eqn:eom_nondim_regroup_pi_1_pi_2} shows that \mstar\ is an independent parameter, the findings show that the system is essentially a function of \massstiff\ and \massdamp. A close inspection of equation \ref{eqn:eom_nondim_regroup_pi_1_pi_2} reveals that \mstar\ only has an impact on the non-linear forcing terms in relation to the velocity of the body. Thus in order for these non-linear terms to be significant, the induced angle of attack and therefore, the velocity of the body needs to be very large, which appears not to be the case for the rage of parameters which were tested. 

 It could be concluded, through comparison between the quasi-steady state and direct numerical simulation data, that the quasi-steady state model provides a good approximation of the power output of the system when \massstiff\ is relatively high. However, the QSS approximation deviates from  DNS predictions at low values of \massstiff\ due to the fact that QSS model does not account for the forces created by vortex shedding which is shown to increase in influence as \massstiff\ is decreased. However, the QSS model does provide a reasonable prediction of the value of \massdamp\ at which maximum power is produced. Both the error in predicted maximum power between the QSS and the DNS models and the relative power of the vortex shedding have been quantified and scale approximately to $1/\sqrt{\massstiff}$  
 
 An expression to describing the frequency of the system was produced from the eigenvalues of the linearised QSS model, in terms of \massstiff\ and \massdamp\. This frequency was defined as the linear frequency \freqlin of the system. Frequency data obtained through this model were compared against QSS model and DNS simulations.
 
 
 The three frequency data obtained through namely, the QSS model, linear frequency and DNS simulations showed a deviation from the undamped natural frequency of the system at $\massstiff<10$. The linear frequency showed a rapid decrease at $\massstiff<1$. It can be concluded that at \massstiff\, where \massstiff\ drops to a significant low level, the non-linear terms of the forcing function of the system start effecting the system. As these non-linearities are not accounted in the linearised QSS model which is used to formulate \freqlin\ a significant deviation of the linear frequency from the QSS frequency could be observed.
 
 The linear frequency agreed well with the DNS results within the boundaries of consideration. The lower boundary of \massstiff\ was limited to $\massstiff=10$ as a clear deviation of \freqlin\ and \freqdns was observed $\massstiff<10$. (figure \ref{fig:pi2-015-freq}). However, as \massstiff\ considered for energy transfer are $\massstiff>10$, it can be concluded that expression formulated for the frequency of the system obtained using the newly formulated parameters \massstiff and \massdamp could be used as a model for prediction of the frequency of an energy harvesting system.  
 
 
 This study concluded the first objective of and the first phase of this research, \emph{``Understanding the governing mechanical parameters of the system and isolate regions where a good power transfer could be obtained."}
 
 The second phase or the objective of this study was focused on optimisation of the governing fluid mechanics of the system in order to obtain a higher power output. The primary hypothesis was that inhibition of the shear layer  reattachment would result in a higher power output. The square cross section was systematically tapered off by changing the \ratio\ ratio in order to achieve this.
 
 
  A negative region in the $\cy$ vs. $\theta$ curve was observed for $\ratio<0.25$. This region resulted in a power loss in a certain portion of the galloping cycle as the driving force $F_y$ and the velocity $\dot{y}$ were in opposite directions.
 
 The mean power versus \massdamp\ curves showed an increase in maximum power as $\ratio$ was decreased until $\ratio=0.25$. At $\ratio=0$,  ($\displaystyle\frac{P_{m}}{\rho \mathcal{A}U^3}=0.0304$ at $\massdamp=0.021$) the maximum power was less than $\ratio=0.25$ ($\displaystyle\frac{P_{m}}{\rho \mathcal{A}U^3}=0.04$ at $\massdamp=0.028$), although the peak value of both the induced angle and \cy\ were greater in $\ratio=0$ compared to $\ratio=0.25$. Further analysis of the \cy\ curve revealed that the negative region of $\ratio=0$  was greater than $\ratio=0.25$, hence resulting in a lower maximum power output. 
 
 The surface pressure plots and the velocity magnitude profiles at the starting points of the wall jets revealed  that there are two mechanisms governing the transverse forcing. The first mechanism is the pressure difference in each shear layer, or the ``streaming effect''. The second mechanism was the relative proximity of the top and bottom shear layers, or the ``proximity effect''.
 
 Initially at $\theta= 4^{\circ}$ the streaming effect dominated resulting the negative \cy. As $\theta$ increased from  $\theta= 16^{\circ}$ to  $\theta= 21^{\circ}$ the proximity effect started dominating resulting in a positive \cy.
 
 Comparison of the QSS and DNS predictions of maximum power showed similar trends. The maximum power increased as \ratio\  decreased supporting the hypothesis of attaining higher power output through inhibition of shear layer reattachment. However, a significant error between the QSS and FSI simulations were observed as \ratio\ was reduced. 
 
 Further investigations carried out using time averaged flow data concluded that the mean flow of FSI simulations had significant deviations from the DNS stationary simulations carried out at corresponding induced angles. This shows that the flow is essentially not quasi-static, violating the primary assumption of considering $F_y$ as the sole driving force of the system. Yet, it can be concluded that the QSS model can be used as a tool to obtain initial approximations to design galloping energy harvesting systems as QSS data similar trends were produced as the FSI simulations.
 
 It was concluded in order to obtain an efficient galloping energy harvesting system through inhibition of shear layer reattachment, one key design consideration is to obtain a cross section which has the optimum balance between the negative and positive regions of the $\cy$ vs. $\theta$ curve. Even though, the inhibition of  the shear layer reattachment through tapering of the trailing edge leads to higher power, as it approaches a triangle, a negative region of $\cy$ emerges in the \cy\ vs. $\theta$ curve. This leads to adverse power transfer. This region keeps increasing between $0\leq\ratio\leq0.25$. Therefore, an optimum \ratio\ should be obtained in order to get a balance between the negative and positive regions which  then leads to an optimal energy transfer. 
 
 As for future research this method of attaining high power through inhibition of shear layer reattachment can be further developed by conducting more detailed investigations into the geometry to find ways to reduce the adverse power transfer which will lead to further increases in power output.
 
The second phase was concluded and the second objective of this study:\emph{ ``Understand the governing fluid mechanics of the system and to optimise and control these mechanics in order to obtain a higher power transfer."} was concluded through this optimisation study of the governing fluid mechanics of the system. 
 
 
 
 