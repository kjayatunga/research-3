\chapter{Conclusions}

A fundamental study was carried out to explore the potential of obtaining useful energy from fluid-elastic galloping. This research was based on numerical models and simulations. The study was primarily focused on understanding the energy transfer between the fluid and the structure.  

As previously mentioned, since  galloping was a fluid-structure mechanism two major objectives were identified for research namely, Understanding the underpinning structural parameters of the system and methods to optimise it for a better power output and understanding the fluid mechanics of the system and thereby obtain an optimum mean power output by controlling these mechanics. A third sub-objective was identified during the study of the first objective, which was to carry out a brief study of the frequency response to compliment the first objective of this research.  

New governing dimensionless groups for galloping namely, \massstiff\ and \massdamp\ were formulated using the natural times-scales of the linearised quasi-steady state model. Data were obtained using a square cross section. The formulated dimensionless groups provided a good collapse for the predicted power output in comparison with the classical VIV parameters which have been traditionally used i.e. \ustar\ and $\zeta$. The collapsed dimensionless groups reinforces the argument the velocity amplitude to the system and the power transfer of the system does not depend on the natural frequency of the system over a large range of natural frequencies. Although equation \ref{eqn:eom_nondim_regroup_pi_1_pi_2} shows that \mstar\ is an independent parameter, the data show that the system is essentially a function of \massstiff\ and \massdamp. A close inspection of equation \ref{eqn:eom_nondim_regroup_pi_1_pi_2} reveals that \mstar\ only has an impact on the non-linear forcing terms in relation to the velocity of the body. Thus in order for these non-linear terms to be acceptable, the the induced angle of attack and therefore, the velocity of the body needs to be very large, which appears not to be the case for the rage of parameters which were tested. 

 It could be concluded through comparison between the quasi-steady state and direct numerical simulation data, that the quasi-steady state model provides a good approximation of the power output of the system when \massstiff\ is relatively high. However, the QSS approximation is to close at low values of \massstiff\ due to the fact that QSS model does not account for the impact of vortex shedding which is shown to increase in influence as \massstiff\ is decreased. Be that as it may, the QSS model does provide a reasonable prediction of the value of \massdamp\ at which maximum power is produced. Both the error in predicted maximum power between the QSS and the DNS models and the relative power of the vortex shedding have been quantified and scale similar to $1/\sqrt{\massstiff}$  
 
 A brief frequency study was carried out in order to complement the understanding of \massstiff\ and \massdamp. Using the eigenvalues of the system an expression for galloping frequency was formulated in terms of \massstiff\ and \massdamp. This frequency was defined as the linear frequency \freqlin\ of the system. Based on this frequency two regions of frequency response was identified which were the linear frequency rage where $f_{lin}$>0 and the non-linear range where $f_{lin}=0$. Both frequency data obtained using QSS model and DNS agreed well with \freqlin\ within the boundaries of the DNS simulations where lower boundary of \massstiff\ was limited to $\massstiff=10$ due to the weakening of the galloping signal.
 
 The QSS frequency data tend to deviate from the linear frequency beyond $\massstiff<10$. Implying the start of the influence of the non-linear forcing. 
 
 The QSS model kept providing a signal and hence a frequency in the non-linear frequency region. Thus, the comparison parameter was the unstamped natural frequency as \freqlin$=0$ and \freqdns could not be obtained from the signal processing techniques used. The data showed and acceptable agreement between $0.06\geq$ but deviated as \massstiff\ reduced.
 
 The mere existence of this non-linear frequency region was a question as no DNS data could be obtained, because the galloping signal was weak and the techniques used to obtain the  frequency was not sensitive enough to capture these weak signals. Thus it was concluded that further investigations should be carried out on this region but was not pursed in this study due to deviation of the major objective and scope and time constrains.
 
 Yet, The linear expression of the galloping frequency formulated using \massstiff\ and \massdamp\ provided a excellent prediction within the boundaries where DNS data were obtained. This complimented the overall understanding and of the new formulated parameters \massstiff\ and \massdamp, completing the first objective of and the first phase of this research, \emph{``Understanding the governing mechanical parameters of the system and isolate regions where a good power transfer could be obtained."}
 
 The second phase or the objective of this study was focused on optimisation of the governing fluid mechanics of the system in order to obtain a higher power output. The primary hypothesis was that delaying the flow re-attachment would lead to a higher power output. The square cross section was systematically tapered off by changing the \ratio\ ratio in order to achieve this.
 
 One interesting observation was the presence of a negative region of the \cy vs. $\theta$ curve beyond $\ratio\leq0$. Thus, a loss of power could be observed in a certain portion of the galloping cycle was present as due to the fact that the velocity and the transverse forcing $F_{y}$ was out of phase. 
 
 It was observed that the maximum mean power increases as \ratio\ was degreased until $\ratio=0.25$. However, further analysis revealed that the maximum power at $\ratio=0.25$ was grater than $\ratio=0$ which was a direct result of the size of the negative region of the \cy\ vs. $\theta$ curve. Thus it could be concluded that the initial hypothesis could be proven but with some conditions.  
 
 Further investigation of the surface pressure data and the velocity magnitude data revealed that the changes in flow velocities at the separation points, as a result of the shape of the cross section and the incidence angle caused this negative region of the \cy plot. 
 
 Comparison of QSS maximum power data and FSI data provided similar trends of maximum power being increased as $\ratio$ was increased proving the initial hypothesis. However, the error between the QSS and FSI maximum power data increased exponentially as \ratio\ reduced. Investigations carried out using time averaged flow-filed data concluded that the mean flow of the FSI simulations had a significant deviation with the corresponding stationary DNS data. This was a result of the incurred higher  traverse velocities as \ratio\ was decreased. AS a result significant non-linear forcing was present resulting a deviation from the quasi-steady hypothesis. Be that as it may, as concluded is phase one of this study QSS model could be used as a tool to obtain initial qualitative approximations to design galloping energy harvesting systems. 
 
 It could be concluded that a key design consideration in obtaining a obtain an optimum cross section for energy harvesting is to find a good balance between the negative and positive regions of the \cy\ vs. $\theta$ curve. Delaying reattachment is beneficial however, the presence of the negative region of the \cy\ curve will have a adverse effect on power transfer. 
 
 Future research and development was also discussed in this phase. A further design considerations could be considered, for example, investigating the possibility of reducing the negative region of the \cy curve by making alterations to the cross section such as rounding the edges of flow separation.
 
 Therefore, the second phase was concluded and the second objective of this study was achieved which was:\emph{ ``Understand the governing fluid mechanics of the system and to optimise and control these mechanics in order to obtain a higher power transfer."}
 
 
 
 