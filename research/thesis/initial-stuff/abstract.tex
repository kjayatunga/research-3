\chapter*{Abstract}


This thesis investigates the potential of energy harvesting through fluid-elastic galloping through studying the energy transfer between the body and the fluid.

This was carried out by numerically integrating a previously derived Quasi-Steady State (QSS), and via Direct Numerical Simulations (DNS) of the fluid-structure system.

A review of the literature identifies a need for new scaling parameters to better represent fluid-elastic galloping. New governing non-dimensional parameters for galloping namely, the combined mass-stiffness, \massstiff\, and the combined mass-damping \massdamp\ are formulated using natural time scales of the linearised QSS model. These new dimensionless groups provide a far better collapse of the predicted power output from the galloping of a square cross section in comparison with the classical parameters, regardless of whether the data comes from the QSS or DNS models. These time scales also provide a linear estimate of the frequency of the system, which is shown to match the frequency measured during the DNS simulations while $\massstiff > 10$.

A comparison between the quasi-steady state and direct numerical simulation data, reveals that the quasi-steady state model provides a good approximation of the power output at high \massstiff. However, the QSS approximation deviates from the DNS predictions at low values of \massstiff\ because the QSS model does not model vortex shedding which becomes more significant as \massstiff\ decreases. However, the QSS model provides a reasonable prediction of the value of \massdamp\ at which maximum power is produced. Both the error in predicted maximum power between the QSS and the DNS models and the relative power of the vortex shedding are quantified and scale approximately to $1/\sqrt{\massstiff}$ .

A semi-empirical search for an optimal body cross section for the extraction of energy is also presented. A hybrid rectangular/triangular body is used, to deliberately test the hypothesis that inhibition of the reattachment of the shear layers can promote large forces, velocities, and therefore energy extraction. It is shown that two features control the energy extraction: the proximity of the shear layer to the body; the velocity of the flow in the shear layers. Both can be controlled by the amount of tapering of the afterbody, and a balance needs to be found between the two to optimize the geometry for energy extraction.

Comparison of results from the QSS and DNS models shows similar trends of maximum power being increased as the body becomes more tapered. However, the difference between the QSS and DNS models increases exponentially as the tapering is increased. Inspection of time averaged flow-field data show that the flow in the true fluid-structure situation is not quasi-static, violating the primary assumption of the QSS model. However, the QSS model still provides a reasonable initial qualitative approximation to design galloping energy harvesting systems. 
