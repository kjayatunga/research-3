\chapter*{Abstract}

The potential of energy harvesting through fluid-elastic galloping is explored through studying the energy transfer between the body and the fluid. This study identified a need for new scaling parameters to better represent fluid-elastic galloping as the parameters used currently (i.e. the traditional Vortex Induced Vibration parameters) were not providing a satisfactory collapse. After reviewing earlier works, this study proposed and tested a hypothesis which assets that inhibition of the shear layer reattachment would lead to a higher power output. To meet the identified requirements, this study was divided into two main phases. Phase one aims to study the underpinning mechanical parameters while phase two was to understand the fluid mechanics of the system and attempt to control the fluid flow to inhibit the shear layer reattachment to gain a higher power output. This fundamental study  was carried out using the Quasi-Steady State (QSS) model and Direct Numerical Simulations (DNS) on both stationary and oscillatory cases are carried out to obtain the data.

New governing non-dimensional parameters for galloping namely, the combined mass-stiffness, \massstiff\, and the combined mass-damping \massdamp were formulated using natural time scales of the linearised QSS model in phase 1. The formulated dimensionless groups provided a good collapse for the predicted power output in comparison with the classical VIV parameters which have been traditionally used, i.e. \ustar, \mstar, and $\zeta$, reinforcing the statement of \citet{Paidoussis2010} that galloping is a ``velocity dependent damping controlled" system.  

A comparison between the quasi-steady state and direct numerical simulation data, revealed that the quasi-steady state model provides a good approximation of the power output at high \massstiff. However, the QSS approximation deviates from the DNS predictions at low values of \massstiff\ because the QSS model does not model vortex shedding which becomes more significant as \massstiff\ decreases. However, the QSS model provided a reasonable prediction of the value of \massdamp\ at which maximum power is produced. Both the error in predicted maximum power between the QSS and the DNS models and the relative power of the vortex shedding have been quantified and scale approximately to $1/\sqrt{\massstiff}$ .
 
An expression for the frequency based on \massstiff\ and \massdamp\ was formulated through the eigenvalues of the linearised QSS model and hence was termed the linear frequency,  \freqlin. Frequency data obtained though this model agreed well with th DNS data  within the boundaries which DNS data were obtained.  

The three frequency data obtained through, the QSS model, linearised QSS model and DNS simulations showed a deviation from the undamped natural frequency of the system at $\massstiff<10$. The linear frequency showed a rapid decrease at $\massstiff<1$. At regions where \massstiff\ drops to a significant low level, the non-linear terms of the forcing function of the system start effecting the system. As these non-linearities are not accounted in the linearised QSS model which is used to formulate \freqlin\ a significant deviation of the linear frequency from the QSS frequency can be observed.

The linear frequency agreed well with the DNS results within the boundaries of consideration. The lower boundary of \massstiff\ was limited to $\massstiff=10$ as a clear deviation of \freqlin\ and \freqdns was observed $\massstiff<10$. However, as \massstiff\ considered for energy transfer are $\massstiff>10$, it was concluded that expression formulated for the frequency of the system obtained using the formulated parameters \massstiff and \massdamp can be used as a model for prediction of the frequency of an energy harvesting system.  

Attaining higher power output through inhibition of the shear layer was the objective of the second phase. This hypothesis was tested out by systematically tapering the sides of the square section parallel to the flow. 

A negative region of the \cy vs. $\theta$ curve beyond $\ratio\leq0.25$ was observed. As a consequence, a loss of power in a certain portion of the galloping cycle was observed which was a result of the direction of the velocity and the transverse forcing $F_{y}$ being out of phase. 

The maximum mean power increases as \ratio\ is decreased until $\ratio=0.25$ which agreed with the hypothesis formulated. However at $\ratio=0.25$ was grater than $\ratio=0$ which was opposite to the outcome expected from shear layer inhibition. Further analysis reavaled that this was due to the size of the negative region of the \cy\ vs. $\theta$ curve.

The initial negative region was created as a result of the uneven flow distribution due to the profile and the positioning of the geometry, which generated \cy\ similar to the generation of lift of an aerofoil. As the incidence angle was further increased this mechanism was suppressed by the force created due to the relative proximity of shear layers to the wall, which was associated with the positive region of the \cy vs. $\theta$ curve. 

Comparison of QSS maximum power data and FSI data provided similar trends of maximum power being increased as $\ratio$ was decreased proving the initial hypothesis. However, the difference between the QSS and FSI maximum power data increased exponentially as \ratio\ reduced. Investigations carried out using time averaged flow-filed data concluded that the mean flow of the FSI simulations had a significant deviation with the corresponding stationary DNS data. This was a result of the incurred higher  traverse velocities as \ratio\ was decreased. As a result significant non-linear forcing was present, which resulted in a deviation from the quasi-steady assumption. However, QSS model can be used as a tool to obtain initial qualitative approximations to design galloping energy harvesting systems. 

Obtaining a good balance between the negative and the positive regions of the of the \cy\ vs. $\theta$ curve is a key design consideration in obtaining an optimum cross section for energy harvesting purposes. Although delaying the shear layer reattachment has it's advantages, the negative region of the \cy\ vs. $\theta$ curve has an adverse effect on power transfer. 








  

