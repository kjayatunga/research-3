\chapter*{Abstract}

The potential of energy harvesting through fluid-elastic galloping is explored through studying the energy transfer between the body and the fluid. This study identified a need for new scaling parameters to better represent fluid-elastic galloping as the parameters used currently (i.e. the traditional Vortex Induced Vibration parameters) were not providing a satisfactory collapse. After reviewing earlier works, this study proposed and tested a hypothesis which assets that delaying the shear layer reattachment would lead to a higher power output. To meet the identified requirements, this study was divided into two main phases. Phase one aims to study the underpinning mechanical parameters while phase two was to understand the fluid mechanics of the system and attempt to control the fluid flow to gain a higher power output.

This fundamental study is carried out using theoretical modelling and numerical simulations. The Quasi-Steady State (QSS) model and Direct Numerical Simulations (DNS) on both stationary and oscillatory cases are carried out to obtain the data.


Phase 1 was initiated by formulating new governing non-dimensional parameters for galloping namely, the combined mass-stiffness, \massstiff\, and the combined mass-damping \massdamp. These parameters were formulated using the natural time scales of the linearised QSS equation. The formulated dimensionless groups provided a good collapse for the predicted power output in comparison with the classical VIV parameters which have been traditionally used, i.e. \ustar, \mstar, and $\zeta$, reinforcing the statement of \citet{Paidoussis2010} that galloping is a ``velocity dependent damping controlled" system. 
A comparison between the quasi-steady state and direct numerical simulation data, revealed that the quasi-steady state model provides a good approximation of the power output at high \massstiff. However, the QSS approximation deviates from the DNS predictions at low values of \massstiff\ because the QSS model does not model vortex shedding which becomes more significant as \massstiff\ decreases. Be that as it may, the QSS model does provide a reasonable prediction of the value of \massdamp\ at which maximum power is produced. Both the error in predicted maximum power between the QSS and the DNS models and the relative power of the vortex shedding have been quantified and scale approximately to $1/\sqrt{\massstiff}$ .

To completely describe the system in terms of \massstiff\ and \massdamp, a frequency study was carried out. An expression for the frequency based on \massstiff\ and \massdamp\ was formulated. This expression was formulated using the eigenvalues of the linearised QSS model and hence was termed the linear frequency,  \freqlin. Two regions of frequency response were identified namely, the region where a linear frequency is predicted and the region where \freqlin\ does not exist. Both frequency data obtained using QSS model and DNS agreed well with \freqlin\ within the boundaries of the DNS simulations conducted, where lower boundary of \massstiff\ was limited to $\massstiff=10$ because the galloping signal got weaker when $\massstiff<10$. 

QSS frequency was scaled with the undamped natural frequency of the system $f$, in the region where a \freqlin\ could not be defined. This revealed that it was within $0.55\leq\frac{f_{QSS}}{f}\leq0.75$ in the rage of $0.06\leq\massstiff\leq0.1$ and dropped further as \massstiff\ reduces.  

The mere existence of this region is questionable as no DNS data could be obtained in this region due to the fact that galloping signal was weak and the techniques used to obtain the  frequency was not sensitive enough to capture the weak signals. There is scope for further study to corroborate or otherwise the QSS prediction using experiments or DNS. 

Be that as it may, The linear expression provided a excellent prediction within the boundaries of data obtained through DNS and therefore confirming that frequency predicted by linearised oscillator model expressed in \massstiff\ and \massdamp\ is accurate.  


The second phase of this study initiated by testing the hypothesis of gaining higher power output by delaying the flow re-attachment. In order to investigate test this, a square cross section was systematically tapered off from the top and bottom of the cross section.

A negative region of the \cy vs. $\theta$ curve beyond $\ratio\leq0.25$ could be observed in this study. Therefore, as a consequence, there is a loss of power in a certain portion of the galloping cycle which is a result of the velocity and the transverse forcing $F_{y}$ being out of phase. 

The maximum mean power increases as \ratio\ is degreased until $\ratio=0.25$. However, further analysis revealed that the maximum power at $\ratio=0.25$ was grater than $\ratio=0$ which was found out to be a direct result of the size of the negative region of the \cy\ vs. $\theta$ curve.

Further investigation of the surface pressure data and the velocity magnitude data revealed that the initial negative region was created as a result of the uneven flow distribution due to the profile and the positioning of the geometry, which generated \cy\ similar to the generation of lift of an aerofoil. As the incidence angle was further increased this mechanism was suppressed by the force created due to the relative proximity of shear layers to the wall, which is typically associated with the positive region of the \cy vs. $\theta$ curve. 

Comparison of QSS maximum power data and FSI data provided similar trends of maximum power being increased as $\ratio$ was decreased proving the initial hypothesis. However, the difference between the QSS and FSI maximum power data increased exponentially as \ratio\ reduced. Investigations carried out using time averaged flow-filed data concluded that the mean flow of the FSI simulations had a significant deviation with the corresponding stationary DNS data. This was a result of the incurred higher  traverse velocities as \ratio\ was decreased. As a result significant non-linear forcing was present, which resulted in a deviation from the quasi-steady assumption. Be that as it may, as concluded in phase one of this study, the QSS model could be used as a tool to obtain initial qualitative approximations to design galloping energy harvesting systems. 

Obtaining a good balance between the negative and the positive regions of the of the \cy\ vs. $\theta$ curve is a key design consideration in obtaining an optimum cross section for energy harvesting purposes. Although delaying the shear layer reattachment has it's advantages, the negative region of the \cy\ vs. $\theta$ curve has an adverse effect on power transfer. 

One of the areas for research in future is investigating ways to reduce the negative region of the \cy\ curve by making alterations to the geometry. Changing the proximity of the shear layers by rounding the edges of the separation points could be one possible approach which can be investigated as future research. 







  

