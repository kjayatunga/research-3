\chapter{Preliminary remarks}

Fluid-structure interactions occurs in many situations in our everyday lives. From the blood flow through our veins to the flight of an A-380 airbus, fluid structure interactions have a significant influence on our lives. On the other hand vibrations are another important phenomenon which have either a desirable or otherwise effect, in mechanical systems.  


Flow induced vibrations are one type of the significant phenomena occurring as a result of fluid structure interactions. In this broader class of flow induced vibrations, fluid-elastic galloping is one commonly visible phenomenon in nature. Fluid-elastic galloping in particular has been widely researched for the past century due to the adverse effects caused on civil structures; where vibrations created through fluid-elastic galloping  leading to failure either through high peak loads or the cumulative effect of fatigue. One such classic example used in the engineering field is the collapse of Tacoma Narrows bridge on November $7^{th}$ 1940. Another example is the vibrations created by galloping on transmission lines due to ice deposition \citep{Parkinson1964}. Hence, due to these adverse effects created by fluid-elastic galloping, extensive research has been conducted to understand its mechanism in order to control and suppress these vibrations.  

With detrimental environmental impact of  fossil fuel and, the search for alternate energy sources with minimal environmental impact has become an important area of research in the modern world, researchers conducting studies on flow induced vibrations are moving towards investigating the possibility of harvesting energy from these vibrations; hence, finding mechanisms to excite and sustain these vibrations\citep{Barrero-Gil2010a}.  

One such research group in University of Michigan has conducted extensive research on energy extraction through Vortex Induced Vibrations (VIV) \citep{Bernitsas2008a-concept, Bernitsas2009, Raghavan2010a, Lee2011b}. However, VIV is a resonance type of phenomenon where the vibrations occur when the vortex shedding frequency aligns with the natural frequency of the system. This phenomenon is known as ``lock-in".

In contrast, fluid-elastic galloping is a ``velocity dependent and damping controlled " mechanism \citep{Paidoussis2010}; thus, operating over a wide range of natural frequencies.  More in-depth discussion on the mechanism of fluid-elastic galloping is presented in section \ref{chap:lit-review}. The fact that galloping operated over a wide range of natural frequencies provides fluid-elastic galloping an advantage over VIV as a mode of energy extraction.

Although extensive research has been conducted in the area of fluid-elastic galloping extensively, the area of energy harvesting through fluid-elastic galloping is quite new where the concept was proposed very recently by \citet{Barrero-Gil2010a}. Thus, more fundamental work is needed in this area, particularly on the energy transfer between the fluid and the body.   
 
To bridge the gap of existing knowledge the following approach has been employed in the work presented in this thesis. A review of literature is presented in chapter 2 where the mechanism of galloping and the theoretical model which describes galloping is extensively discussed with reference of existing literature; as well as the  gaps of current knowledge on energy transfer during galloping are identified. Based on these identifications of the gaps of the current knowledge, the objectives are defined.

The study is presented in two phases. Phase 1 is focused on understanding the governing mechanical parameters followed by phase 2 where the possibility of achieving a higher power output though inhibition of shear layer reattachment is investigated.  

The tools employed to carry out this study are discussed in chapter 3, where the methodology and validation are presented. Here, the quasi-steady state model is introduced and the method of numerical integration in order to solve this model is discussed followed by the presentation of equations which are used to calculate average power. Direct Numerical Simulations(DNS) at low Reynolds numbers are carried out for both stationary and  oscillating bluff body. The models and numerical algorithms employed to carry out the DNS are presented, followed by a convergence and validation study.  

As a lack of suitable for scaling parameters to describe galloping is identified in the literature review; a new set of non-dimensionalised scaling parameters namely \massstiff\ and \massdamp\ formulated from the linearised Quasi-Steady State (QSS) model and presented in chapter 4. These parameters are then compared with the existing scaling parameters. The influence of these parameters on mean power is then discussed in \massstiff\ and \massdamp\ space followed by a comparison between the QSS and DNS data.

The influence of \massstiff\ and \massdamp\ on fluid-elastic galloping is further investigated in this chapter through a study on the influence of the new scaling parameters on the frequency response. An expression for the galloping frequency is formulated based on \massstiff\ and \massdamp\ using the eigenvalues of the system. The frequency data obtained from this model are compared with data obtained using other approaches. The limitations of this linear frequency model are identified and the region where this model could be applied are identified and quantified.

The results and discussion on the work carried out on phase 2 are presented in chapter 5. As was hypothesised that inhibition of the shear layer re-attachment could lead to higher power output based on the data presented in \citet{Luo1994}, the testing of this hypothesis is carried out here. 

The shear layer re-attachment is inhibited systematically by tapering away the top and bottom trailing edges of the square cross section. The static body results, QSS predictions, the predictions from the fluid-structure interaction simulations and the underpinning fluid-mechanics are discussed. This chapter concludes with presentation of some fundamental design considerations to be used to obtain an efficient energy harvesting system through control of the shear layer reattachment.

Finally, the conclusions obtained from this study are presented in chapter 6.









