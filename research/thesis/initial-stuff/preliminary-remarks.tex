\chapter{Preliminary remarks}

Fluid-structure interactions surrounds the whole ecosystem we live in. From the blood flow through our veins to the flight of an A-380 airbus, fluid structure interactions have a significant influence in our lives. On the other hand vibrations are another important phenomenon which governs the whole ecosystem. From the process of our respiratory cycle to the facebook status update we put vibrations influences every aspect of our lives and even life as a whole.  


Flow induced vibrations are one of the significant phenomenon occurred as a result of fluid structure interactions. In this broader class of flow induced vibrations, fluid-elastic galloping is one commonly visible phenomenon in nature. Fluid-elastic galloping in particular has been widely researched for the past century due to the adversed effects caused on civil structures; where vibrations crated through fluid-elastic galloping  leading to failure either through high peak loads or the cumulative effect of fatigue. One such classic example used in the mechanical engineering filed is the collapse of Tacoma Narrows bridge on November $7^{th}$ 1940. Another example is the vibrations crated by galloping on transmission lines due to ice deposition \citep{Parkinson1964}. Hence, due to these adverse effects crated by fluid-elastic galloping, extensive research were conducted to understand this mechanism in order to suppress these vibrations.  

With exponential depreciation of  fossil fuel and due to its adverse impact on the environment such as global warming, the search for alternate energy sources with minimal environmental impact has become an important area of research in the modern world. Thus, researchers conducting studies on flow induced vibrations are moving towards investigating the possibility of converting these adverse vibrations to a mode of energy harvesting; hence, finding the mechanisms to excite and sustain these vibrations\citep{Barrero-Gil2010a}.  

One such research group in university of Michigan have conducted extensive research on energy extraction through Vortex Induced Vibrations (VIV) \citep{Bernitsas2008a-concept, Bernitsas2009, Raghavan2010a, Lee2011b}. However, VIV is resonance type of phenomenon where the vibrations occur when the vortex shedding frequency align with the natural frequency of the system also knows as ``lock-in".

In contrast, fluid-elastic galloping is a ``velocity dependent and damping controlled " mechanism \citep{Paidoussis2010}; thus, operating over range of natural frequencies. This fact provides fluid-elastic galloping an advantage over VIV as a mode of energy extraction. 

Although extensive research has been conducted in the area of fluid-elastic galloping extensively, the area of energy harvesting through fluid-elastic galloping is quite new where the concept was placed very recently by \citet{Barrero-Gil2010a}. Thus, much fundamental work is needed in this area particularly on the energy transfer between the fluid and the body.   
 
To bridge the gap of existing knowledge the following approach has been employed in the work presented in this thesis. A review of literature is presented in chapter 2 to identify the gaps of current knowledge on energy transfer during galloping. Based on these identifications, the objectives are defined. The study is presented in two phases which are understanding the governing mechanical parameters and the governing fluid dynamics of the system in order to obtain high power output.  

The tools employed to carry out this study are discussed in chapter 3, where the methodology and validation are presented. Here, the quasi-steady state model is introduced and the method of numerical integration in order to solve this model is discussed followed by the presentation of equations which are used to calculate average power. Direct Numerical Simulations(DNS) at low Reynolds numbers are carried out in both stationary and  oscillatory form. Thus, the models and numerical algorithms employed to carry out the DNS are presented, followed by a set of convergence and validation studies.  

A need for a better scaling parameters is identified in the literature review. Thus, a new set of non-dimensionalised scaling parameters namely \massstiff\ and \massdamp\ are formulated from the linearised Quasi-Steady State (QSS) model are presented in chapter 4. These parameters are then compared with the existing scaling parameters. The influence of these parameters on mean power is then discussed in \massstiff\ and \massdamp\ space followed by a comparison between the QSS and DNS data.

The influence of \massstiff\ and \massdamp\ in fluid-elastic galloping is further studied in chapter 5, through an investigation on the influence of the new scaling parameters on the frequency response. A expression for the galloping frequency is formulated based on \massstiff\ and \massdamp\ using the eigenvalues of the system. The frequency data obtained from this model are compared with data obtained using other approaches. The limitations of this linear frequency model are identified and the region where this model could be applied are identified and quantified.

The tasks are identified for the second phase of this study are discussed in chapter 6. It was hypothesised that a delay in shear layer re-attachment could lead to higher power output based on the data presented in \citet{Luo1994}. This hypothesis is tested here. The shear layer re-attachment is delayed by systematically tapering away the top and bottom walls of the square cross section. The static body results, QSS predictions, the predictions from the fluid-structure interaction simulations and the underpinning fluid-mechanics are discussed. This chapter concludes with presentation of some fundamental design considerations to be used to obtain an efficient energy harvesting system through control of the governing fluid mechanics.

Finally, The conclusions obtained from this study are presented in chapter 7, which were based on the formulated objectives of this research.  

      









