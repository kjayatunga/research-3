\chapter{Preliminary remarks}

Fluid-structure interactions surrounds the whole ecosystem we live in. From the blood flow through our veins to the flight of an A-380 airbus, fluid structure interactions have a significant influence in our lives. On the other hand vibrations are another important phenomenon which governs the whole ecosystem. From the process of our respiratory cycle to the facebook status update we put vibrations influences every aspect of our lives and even life as a whole.  


Flow induced vibrations are one of the significant phenomenon occurred as a result of fluid structure interactions. In this broader class of flow induced vibrations, fluid-elastic galloping is one commonly visible phenomenon in nature. Fluid-elastic galloping in particular has been widely researched for the past century due to the adversed effects caused on civil structures; where vibrations crated through fluid-elastic galloping  leading to failure either through high peak loads or the cumulative effect of fatigue. One such classic example used in the mechanical engineering filed is the collapse of Tacoma Narrows bridge on November $7^{th}$ 1940. Another example is the vibrations crated by galloping on transmission lines due to ice deposition \citep{Parkinson1964}. Hence, due to these adverse effects crated by fluid-elastic galloping, extensive research were conducted to understand this mechanism in order to suppress these vibrations.  

With exponential depreciation of  fossil fuel and due to its adverse impact on the environment such as global warming, the search for alternate energy sources with minimal environmental impact has become an important area of research in the modern world. Thus, researchers conducting studies on flow induced vibrations are moving towards investigating the possibility of converting these adverse vibrations to a mode of energy harvesting; hence, finding the mechanisms to excite and sustain these vibrations\citep{Barrero-Gil2010a}.  

One such research group in university of Michigan have conducted extensive research on energy extraction through Vortex Induced Vibrations (VIV) \citep{Bernitsas2008a-concept, Bernitsas2009, Raghavan2010a, Lee2011b}. However, VIV is resonance type of phenomenon where the vibrations occur when the vortex shedding frequency align with the natural frequency of the system also knows as ``lock-in".

In contrast, fluid-elastic galloping is a ``velocity dependent and damping controlled " mechanism \citep{Paidoussis2010}; thus, operating over range of natural frequencies. This fact provides fluid-elastic galloping an advantage over VIV as a mode of energy extraction. 

Although extensive research has been conducted in the are of fluid-elastic galloping extensively, the area of energy harvesting through fluid-elastic galloping is quite new where the concept was place very recently by \citet{Barrero-Gil2010a}. Thus, much fundamental work is needed in this area particularly on the energy transfer from   






